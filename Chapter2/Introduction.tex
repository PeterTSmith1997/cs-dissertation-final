\section*{Introduction}
This chapter will look at available literature that focuses on various types of attacks that require low amounts of bandwidth. As part of this section, port scanning attacks will also be addressed and studied with supporting literature. In general these types of attack are much more difficult to detect that high bandwidth attacks, and an evolving strategy of stealth has been adopted by attackers over the years. It should be noted that port scanning attacks in general could be broken down into many different sub categories. A detailed breakdown of these subcategories shall be presented in section 2.2.1. Due to the fundamental core architecture of port scanning attacks, the literature review shall cover the attack definition as a whole, rather than addressing each subcategory.

In order to approach the topic of low rate and port scanning attacks some basic understanding of terminology should be addressed. In general these types of attack rely upon a weaknesses in protocol such as the transmission control protocol (TCP). A TCP is a connection-oriented protocol, a connection needs to be established before two devices can communicate. TCP uses a process called three-way handshake to negotiate the sequence and acknowledgment fields and start the session. Two devices would be in communication, the Client and the Server. The Client initiates the connection by sending the TCP SYN (SYNchronize) packet to the destination Server and the Server receives the packet and responds with an acknowledgement. The Client then acknowledges the response of the Server by sending the acknowledgment back, and a connection is formed. 
It is also important to understand the current web-server protocols and their differences in terms of security. It is the initial assumption that there are differing vulnerabilities for HHTP/2 and HTTP/1.1 web-servers. HTTP/2 (defined as RFC 7540), is a more modern protocol, than that of HTTP/1.1 (defined as RFC 2616). It is assumed that due to the recent implementation of this protocol that it has more vulnerabilities through a wider range of threat vectors. This shall be investigated and hopefully illustrated through the literature available for study in this chapter. The initial ideology behind this theory was formed when, Tripathi suggested that HTTP/2 has more threat vectors than HTTP/1.1 (\cite{tripathi2018slow}). Adi also notes that "The HTTP/2-standard states that if the host machine does not monitor resource usage, it exposes itself to the risk of a DoS attack" (\cite{Adi2015}).

The comprehension of these protocols are key in the greater understanding of how both low rate bandwidth attacks and port scanning attacks operate. Both of these attacks will actively exploit protocols to carry out their attack. 