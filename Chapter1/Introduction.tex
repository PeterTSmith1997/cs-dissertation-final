
As mankind heads into the twenty first century, the internet has become a critical part of the infrastructure for everyday life. As such, it has developed into a vital economic asset for every country. Many aspects of modern society are now switching towards the use of internet, either partially or exclusively. This creates a delicate check in the vein which can be exploited by bad actors, such as malicious foreign governments or criminal organisations, looking to deny availability to websites for ransom or political gain. An infamous example of an attack for political gain was carried out during the UK general election of 2019. The Labour Party was hit by a two stage cyber-attack that attempted to cripple their website during this critical time (\cite{Labour}). Labour commented that these attacks originated from computers in Russia and Brazil and attempted to flood their servers with illegitimate traffic. Fortunately many of the affects of the attacks were mitigated by their use of Cloudflare's defensive mechanisms, which shall be discussed in depth during this thesis.


As the internet grows and becomes increasingly more reliant upon richer content, through social media and video, it has become apparent that HTTP/1.1 is becoming outdated and substandard to functionality.Its replacement, HTTP/2 was found to have a host of security limitations, which will be discussed later on in this thesis. Due to the widespread adoption of HTTP/2 over the last couple of years as a replacement for HTTP/1, research needs to be conducted into the safety of HTTP/2, the conclusions from the 2016 study by Erwin Adi suggested that the HTTP/2 protocol does not restrict the intensity of traffic generated. This highlighted the need for further research into the protocol itself. A potential application of a further protocol or mechanism with the aim of identifying the volumes and patterns of network traffic communicated between a client and network machine may be required. After consideration of the conclusions from the Erwin Adi studies it was decided to investigate the potential of creating an early detection methodology. A consideration of using the HTTP/2 format here is that during the writing of this report HTTP 3 became available to a limited number of websites, however, due to how new this protocol is, there is a lack of research available in order to assess the protocol within this work.


Due to the ability to automate these malicious assaults, it is not just large organisations that need to feel the danger of attacks. These high rate bandwidth attacks overload the server with false communication, hence denying it to legitimate users by what is known as a denial of service attack (DoS). A report from NETSCOUT has estimated that Distributed Denial of Server (DDoS) attacks alone could be costing the UK economy more than £1 billion a year due to the damage being done. The report estimated that the average cost for each UK business that had seen downtime due to DDoS attacks exceeded £140,000 (\cite{Costs}).

The use of real time attack prevention software such as that offered by Cloudflare is widely accepted and used by businesses and organisations. There is an abundance of software that is available at a commercial level for high rate bandwidth protection. This thesis evaluates why high rate DoS attacks are easier to detect and mitigate, however, these are normally used by large scale hosting providers and other content delivery network providers. Whilst these are normally very effective at blocking such attacks, there are some inherent flaws with these types of protection methods. 

\section{Topic of Research}

This research looks at some of these industrial approaches to current protection methodology, however, the research also considers past studies into the methodology of detection for Low rate and port scanning attacks. High rate bandwidth attacks are reviewed with regards to the ease of identification, and identify the reasons why. A critical evaluation of current and historical techniques has been conducted in order to establish an outline for best practice in attack detection. It was the the directive of the research to create a formulaic program to assess the risk of incoming traffic, that is resource effective and user friendly. The planned architecture of the formula was purposed to be able to accurately identify a variety of attacks. The formula also needed to be generic enough to be used on a variety of websites.

The first is that these systems mainly rely on real-time data and traffic; this data is aggregated, therefore, what may be an attack on one website could potentially not be flagged on another. This is because, to the overall network, what looks like a normal traffic pattern on one website may in fact be malicious on another, due to the aggregated nature of the data. Current systems do not provide a lot of information to the end users; this may increase the risk factor to their website making it less safe and giving them a less accurate picture of what is going on. For example, they may see more traffic than normal, however, they may not be informed if there is a legitimate traffic spike or a malicious attack. An attack may be stopped; this may make the website owner complacent in regards to security. The software proposed will try to actively engage the users in their security.

The research had theorised that the data contained within log files would be sufficient to diagnose malicious trends of incoming traffic. As the information stored within the log files is particularly meagre, the research has attempted to utilise as many aspects as possible from the raw data to draw conclusions regarding risk related elements. After studying the log file data in greater depth, the risk factors have been given a scoring system in order to formulate a mathematical approach to aid in the detection of malicious traffic. 

As it could be the case that this software might be used by individuals with limited PC awareness, this thesis also consults historical studies into human computer interaction. In particular this work looks at the ability of non-cyber security experts in identifying attacks. With this knowledge, an appropriate user interface has been constructed with the principle of working hand in hand with the software to identity low-rate bandwidth attacks. The lack of concrete research on Low-bandwidth attacks that are able to work in a production environment is a concern to the safety of those on the internet, therefore, this work will propose new ways of detecting attacks that are able to work in a production environment.

\section{Scope of Work}

It is important to set out the framework of the scope of research in terms of what is deemed to be achievable within the window of the project. It is the intention of the thesis to build a system for identifying and mitigating malicious traffic. This may include a mixture of a formulaic approach, AI or a human moderator. After the literature review is conducted, best practices in methodology shall be adopted using the lessons learned by prior researchers into the field.

Due to the varying nature of attacks that can take place online, particularly on a website, any formula or AI proposed for construction in this thesis shall be limited to the detection of common attacks. The system shall take a 'broad stroke' approach to detection methodology. However, it is theorised that attackers are using a constantly evolving stealth strategy. Therefore it is reasonable to conclude that the software may not be able to detect every type of attack with a high degree of accuracy.

A user interface shall be designed to work hand in hand with the developed software. Whether or not this user interface is purely intended to run the program, or used in the process of threat detection, user testing shall be carried out to promote ease of use and user understanding. Although the user interface shall be functional and practical, it is not the intention to produce a user interface suitable for commercial release. This may mean that the user interface will be less visually impressive than standards expected for a marketable product.
