%!TeX root=Dissertation.tex

\section{What is the purpose of this study?}
I have developed a small application to aid website owners to detect attacks on their website. This study aims to look at whether the software is usable for website owners. It is important to find out if website owners can use the software and understand the output of the data. I am conducting this study as part of my degree at Northumbria University. 
\section{Why have I been invited?}
You have invited to take part in this study as you are website owner or, you have the technical knowledge in web technology. 
\section{Do I have to take part?}
You do not have to take part in the study, you can stop at anytime. I am giving you this Participant Information Sheet to allow you to make an informed decision. You are fully able to decide whether or not you want to take part. 

\section{What will happen if I take part?}
You will be sat in front of a computer for about twenty minutes and will be asked to perform various tasks within the software. Your actions and comments will be captured using a screen recording software, you will be asked to think aloud and explain the reasons behind your actions on the software as well as what you are trying to achieve. After you have completed the task, you will be asked some general questions about your experience of using the software, this will ensure we get the same data from each participant. Your name will not be recorded on any information sheet or during the screen recording. The consent form that you will sign will not be stored with other data. On the questionnaire you will be given an I.D number which will be used to identify your recording so that we can check your questionnaire answers along with your recording. 
\section{How will my data stored and how long will it be stored for?}
Your paper consent form will be virtualised immediately upon completion and the original will be destroyed. The questionnaires will be completed online and the data will be retrieved from the online system, once all data is collected it will be deleted from the online system and stored in a password protected document on the university protected U-Drive. The videos will be stored on the university U-Drive and be destroyed after the work has been marked.  All data will be handled in accordance with the Data Protection Act (2018).
\section{What categories of personal data will be collected and processed in this study?}
Your name will be collected so that we can ensure we have your informed consent, as part of the data collection process we will be required to record your voice, this enables us to understand what users are thinking when using our software. 
\section{What is the legal basis for processing data?}
Based on the nature of the study, the data will be collected in accordance to Article 6(1)(e) "processing is necessary for the performance of a task carried out in the public interest". This study will collect a recording of your voice, due to this being sensitive personal data we will also be collecting data in accordance with Article 9 (2)(j) “processing is necessary for scientific and historical research purposes".
\section{Who are the recipients or categories of recipients of personal data, if any?}
No other external party will have access to the data provided during this study.
\section{What will happen to the results of the study and could personal data collected be used in future research?}
Once the dissertation has been marked, all personal data is deleted, any findings may be referenced in future research and the dissertation may be published.
\section{Who is Organizing and Funding the Study?}
This study is organised by Northumbria University. 
\section{Who has reviewed this study?}
The research project, submission reference 18619 has been approved in Northumbria University’s Ethics Online system. It has been reviewed in order to safeguard your interests, and have granted approval to conduct the study.
\section{What are my rights as a participant in this study?}
As a participant, you have the right to access a copy of the information collated in their personal data (to request access participants are required to submit a Subject Access Request); this can be done through email. If the participants notices that there is innacurate documentation of their personal data they have the right to request this to be corrected. You are also informed that if you are dissatisfied with the University’s processing of personal data, you have the right to complain to the Information Commissioner’s Office. 
\bigskip
\begin{center}\textbf{
    Contact for further information: }
    
Researcher Name: Peter Smith
Researcher email: peter.t.smith@northumbria.ac.uk
 
Supervisor: Nick Dalton 
Supervisor email: nick.dalton@northumbria.ac.uk

Second Marker: Neil Elliot
Second Marker email: neil.elliot@northumbria.ac.uk

Name and contact details of the Records and Information Officer at Northumbria University: Duncan James (dp.officer@northumbria.ac.uk). 

You can find out more about how we use your information at: www.northumbria.ac.uk/about-us/leadership-governance/vice-chancellors-office/legal-services-team/gdpr/gdpr---privacy-notices/ 
or by contacting a member of the research team

\end{center}