\section{CDNs and Proxies}
\subsection{Introduction}
When it come to DDoS attack detection there are many major proxies available, this report shall focus on two of the market leaders, Cloudflare and Akamai, as they have particularly different marketing strategies. Comparisons in defence mitigation and detection techniques will be discussed in this section. It must be noted, that due to the competitive nature of this industry, in depth techniques may not be available for discussion due to corporate confidentiality. One thing is apparent when reading the literature from the previous section; high rate and low rate attacks are better dealt with at a proxy level.
\subsection{Cloudflare}
Cloudflare have structured their marketing strategy around a freemium model, where some elements of the service are offered for free; this means that all websites can have some level of protection against attack. It is affordable to maintain Cloudflare's uptime when compared to Akamai who charge clients by a data transfer rate. Cloudflare utilises a networking approach called anycast; this allows them the unique ability to route paths to two or more endpoint destinations. Routers will select the desired path on the basis of a number of hops, distance, lowest cost, latency measurements or based on the least congested route. Allowing for them to essentially mitigate a high rate bandwidth attack by distributing attack traffic across multiple data centres and soaking the impact of incoming attacks. Cloudflare has also been active in addressing the numerous vulnerabilities in the HTTP/2 protocol that have been identified and discussed in prior chapters. This will help to reduce the number of threat vectors from which which numerous different types of attacks can be launched (\cite{CFHTTP2}).

Cloudflare has a history of actively participating in DDoS simulations in order to test its defence methodology and mitigation techniques. One particular simulation was carried out at their office in Austin Texas where software engineer employees were sent home then instructed to launch random high rate attacks against Cloudflare's Texas office network. The initial effects were felt and staff members on site reported a degradation in audio and video quality for online tasking, however, after Gatebot, (another defence mechanism used by Cloudflare), was applied, the quality of all services was fully restored(\cite{CFAustin}). Gatebot has been discussed in chapter 2 during the literature review of low bandwidth attacks. The results showed that they were able to identify and mitigate the attack, they even tested a new system of networking protection called magic transit which was successful in actually accelerating genuine traffic during this attack phase.

The power of Cloudflare would appear to be down to the number of websites from which the CDN can extract traffic patterns. This is encouraged by their freemium model which stimulates a larger sign up volume of a large range of websites world wide. The data on malicious IP activity is invaluable to them in order to advance the product features to filter out a wide array of known hostile IPs. The Cloudflare website boasts an impressive 20 million internet properties observing 1 billion unique IP addresses every day.

Cloudflare's approach appears to be one of openness and transparency which is beneficial to the evolution of security on the internet, however, Akamai appears to be taking a different approach. 
\subsection{Akamai}
Akamai are perhaps the earliest example of a CDN model and are widely considered the pioneers in the industry. Their pricing structure is not widely advertised, however when contacted in relation to their products and pricing, they responded that the usual process is to scope out an individual business's requirements.\cite{akamai} Akamai does however advertise a free 30 day trial to encourage businesses to test their defense systems with some terms and conditions which are unavailable to view without establishing the package of protection. In 2016, a free trial client was dropped during a 620 GB/Second DDoS attack suggesting that they were unwilling to mitigate this level of attack to a free user during a trial period. When compared to the Githhub attack of 2018, this was relatively small in comparison.

Akamai's protection methodology is based around the placement of 'scrubbing centres' around the world to which spike traffic can be routed in order to mitigate high rate attacks. Although the actual capabilities of these centres is not officially disclosed, a vice president of the company spoke to wired in 2018 and gave a hint to Akamai's capabilities. Josh Shaul VP suggested that Akamai based their scrubbing capacity around 5 times the size of the highest Rate per second attack in history which peaked at 1.2 TB/Second(\cite{github}). This approach does not account for the potential for multiple attacks that may take place at the same time on the servers of multiple customers.

In summary Akamai seems to have a wavering market share hold, and the apparent shrinkage of their data scrubbing centres in the past few years is a clear sign of their loss of customers to other CDNs. This may lead to a loss of service during an attack to their protectorates if simultaneous attacks do take place due to the ever increasing severity of high rate bandwidth attacks. Cloudflare appear to have a much healthier and sustainable business model. It is self invigorating, mostly due to the huge numbers of freemium protectorates, and the data from which they can harvest to develop future defence mechanisms. 