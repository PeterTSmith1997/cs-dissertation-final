\section{Content Management Systems (CMS)}

There are a wide range of CMS, or content management systems available online. Many website owners use content management systems in order to store and modify digital content. One particular high profile CMS is WordPress. At time of writing, WordPress holds around 62\% of the CMS market share (\cite{Wordpress}). WordPress already has a lot of plugins available for security, and whilst these do a good job at blocking attacks in real-time and preventing invalid logins they are not able to look at data over longer periods, therefore some of the long and slow attacks could go undetected due to the limitations in the length of time that WordFence and other similar plugins analyse the data for. According to their website, WordFence includes an endpoint firewall and malware scanner to protect WordPress from malicious activity. 

Due to the fact that WordPress is so popular there is an inherent risk of WordPress style attacks on all websites. Attackers will use a common vector of attack technique to attempt to access a WordPress login page. Due to this it would be appropriate for any build of defensive software to apply a risk related scoring to any attempts to directly access a \textit{"/WP-admin"} page. However, they should be a low risk factor as it is difficult to distinguish between an attacker or a genuine administrator.

A lot of the plugins such as WordFence automatically block IPs that match certain patterns, for example, if they generate a lot of errors. As shall be discussed in the next chapter this may be a weakness of WordFence, due largely to discussion about whether human input should be removed when making security decisions.

\section{Summary}
Throughout this work so far it has become apparent that there is no one sure way of defending a website from any type of attack. Instead defences should be combined in a multi layered way in order to attempt to stop attacks. Another point that has become apparent is the fact that the more data that can be collected about network activity the easier it is to identify malicious activity. It is also worth noting that by sharing data about good company practices in mitigating attacks, the development of attack prevention will increase ensuring an increase in safety for website owners.