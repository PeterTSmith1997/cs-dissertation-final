\section{UI design} \label{ui}

As previously discussed in Chapter \ref{Human factors} it was clear that the user interface is a vital component in the process of accurate identification of malicious traffic. It was illustrated by prior research that novice users with little or no knowledge of Cyber security have a tough time identifying malicious activity. However, it is possible for them to do so given the correct information in a clear and self explanatory format. Given the conclusions from research that observed novice users when approaching the detection of threats, it was good practice to apply the utilisation of Nielson's principals. Consulting these principals aided in the construction of a user interface that promotes understanding and ease of functionality.

A minimalistic design was chosen for the creation of the UI. One of Nielson's 10 heuristics for user interface design warned that 'Dialogues should not contain information which is irrelevant or rarely needed.' For this reason a UI based around functionality over aesthetics was chosen as a fundamental for design. The UI is particularly basic, however, it displays the relevant data to the user and is highly functional. The overall log file data is broken down into three distinct categories with clear and self explanatory headers supporting the usability for non expert users.

To aid the user experience, when a window is closed, the software goes back to the previous window. Each of the two main UI sections are centred around a main interface. The website owner interface is centred around the data from their log-files and when they click on one, more detailed data is shown, this includes the calculated risk factor with a coloured bar: coloured green below 25\%, yellow between 26-50\%, orange between 51-75\% and red 76\% and above; this gives a very quick visual indication of the risk associated with the IP being looked at. 

Within the UI the decision was made to use clear and simple English and avoid the use of codes and complex terminology. Nielson suggests that a designed system should speak the 'users' language', with words, phrases and concepts familiar to the user, rather than system-oriented terms, that follow real-world conventions, making information appear in a natural and logical order. This approach is supportive to novice users and aids in their understanding of the software.

The user interface was constructed with data aligned to clear columns and lines in order to keep the format neat and tidy, and draw the users eye to appropriate data fields. \citeauthor{cranor2008framework} stated that the implementation of system features that are intuitive, and simple to use will help to support a novice to engage with them (\cite{cranor2008framework}).

The user interface that shows the risk of an IP is made up of two separate tabs. The first tab shows a high level summary of the IP including: Times visited, total data sent, country code, times reported, reports in the last 30 days and agent or bot status. These data fields were chosen as they give the user a very quick and easy way to see if an IP is malicious. The second tab provides a more in-depth view of the raw log file data, it was important to include this for more advanced users who may wish to interact with the raw data. Even though the data is presented in the same way as the log file, it could be argued that it is easier to understand as it is only showing data from an IP rather than all IP addresses. IT is important to note that the risk factor is displayed on both screens, so the user is not disadvantaged for using either tab.

Nielson stated that Flexibility and efficiency of use is an important factor when designing a UI. Adhering to this principal would support the usability of both basic and advanced features. This is an imperative methodology of design which allows for novice users to use the basic fundamentals of a UI, while allowing experts to learn and excel. While designing the UI, the decision was made to add an internal tab for full details of an incoming IP. This, may be overwhelming for a novice user, however, an expert user could derive much more information regarding the instance of incoming traffic from the use of this feature. Even though this is still raw log file data, it could be argued that it is easier to interpret than the raw log file, as it only pertains to a single IP.


The implementation of Error messages has been applied to the user interface. These messages are generic to maintain the integrity of the software. They are simple to read and understand for even a novice user and help users recognize, diagnose, and recover from errors within the software. This is in line with Nielsons principal for self explanatory Error message presentation. Nielson reiterates the importance of a user interface to support Error prevention. Even better than good error messages is a careful design which prevents a problem from occurring in the first place. The user interface has adhered to this principal as when an IP has been previously listed as a bot, the software has been built to prevent the user from reporting this as a bot a second time.

