\subsection{Limitations of work}

Compared to other work, as outlined in chapter 2,3 and 4, there was a lack of knowledge about using log files to identify attacks  to build upon and therefore, this work will have some inherent limitations. There were a number of factors that had to be decided without the insight from any prior literature. An example of this is when assessing the risk of an attack by country, the formula only looks at the number of attacks from that particular country. A better way of calculating this risk might be, to look at the number of attacks per head of population; this would be more suitable as the current method is penalising to large countries based on the calculation. The calculation for doing this per head would have been overly complicated thus was not taken at this time for reasons of time constraint. It is believed that including this feature would be very little additional benefit to the software at this stage.

During the research, questions regarding the identification of the true origin of IP addresses were raised as a concern. It would be prudent due to the increasing use of VPN activity online, to explore the possibilities for alterations to the software in order to compensate for this factor. A possible workaround for this may be through the addition of a minor alteration to this project. This would entail an extension of the database to include known VPN IP addresses, much like the recording of known bots. However, as the software stands, administrators could still flag IPs as a bot without the instance being categorised as a VPN. This may be an approach to deal with such traffic instances before any larger patch implementation.

After the collection of the dataset, it was discovered that there were some limitations with the volume of traffic that was collected. Upon reflection, a larger pool of network traffic may have been instrumental in assessing the ability of the formula to detect different types of malicious attack. It was concluded that the dataset did include a few instances of malicious activity which was fortunate as this showed the formula in action when addressing malicious IP instances.

During the analysis of the raw data it was theorised that an administrator of a website was falsely identified as a malicious IP by the software, due to risk related flagging. Appropriate adaptations should be implemented to the software in order to mitigate this potential weakness. A reasonable workaround for this would be the ability for every user to log their own IP within the software or UI. This would enable the software to ignore their own IP address when the log files are assessed for malicious activity.

Another potential failing of the software is that the dates on which the IP was reported are not necessarily the dates of the attack. This may flag up malign, historical IP addresses if the website owner identifies an old IP with suspicious trends. This factor would only come into play if an IP address becomes reformed or reused by a non malicious user. For this reason it is imagined that this would only be a small limiting factor.

When looking at the risk of an individual country, the values used in the software are only based on the number of attacks per country. While this is a good way to assess the risk of a country, this methodology could potentially have issues, for example, larger countries will statistically have more attacks than smaller countries. According to the Nexusguard 2018 2 threat report, it was identified that 23.34\% of attacks came from China, with a further 14.90\% coming from the USA; this is said to be expected, due to the internet presence of their citizens parentage per population, with China's being over one billion users. Therefore, it would be good to look at the total number of attacks per country that are reported to the software for instance, compared to the size of the population. There is little reason why the measure described in this paragraph could not already be used, as an admin user can manually enter the risk of the country. Therefore, the proper implementation of this feature could bring an increased accuracy rating of the software. 

It could be argued that another limitation of the work is that the software does not automatically report suspicious IPs. This is all left down to the user, therefore, it could in theory be argued that the data may be incomplete, thus, any security advice given to other users could be deemed to be inaccurate. The implementation of an automatic report system would be beneficial if certain criteria were met, If these IP addresses were potential bots, then reporting these IPs and marking them as bots would improve the the system. Additionally, if the IP addresses were above a certain risk score, they could be automatically reported to help identify malicious traffic. Other features could also be looked at to automatically report these IP addresses.

After the literature review was completed the required methodology for user testing was becoming increasingly complex to conduct. It became apparent through Ben Asher's work that conducting user testing thoroughly would be time consuming, and would require many user groups for participation. After the outbreak of Covid-19, the decision was made to terminate all plans for user testing as the ability to communicate with participants in a controlled environment was impossible. For this reason a limitation of a fully tested user interface should be illustrated. 