\section{Legal, Social and Professional Issues}

During the research phase of my project several barriers were encountered that impeded my analysis, the majority of which stemmed from the lack of transparency from cyber security companies. One such barrier was encountered during email correspondence with the company Akamai. I reached out to the company and established a professional dialogue by email in an attempt to collect information regarding the products and services that they offered. After disclosing the fact that I was a student of Northumbria University the company representative closed the process of communication. It was believed that this was due to the fact that the correspondent was only interested in the sale of products. However, I remained professional throughout the correspondence and made relevant and honest disclosures when appropriate.

At one stage within the literature review I made direct contact with a key researcher into DDoS attacks and detection methodology \citeauthor{Adi2015}. I found this to be a particularly rewarding experience as the researcher was very happy to share further information regarding his research methodology during both of his studies within the literature review. This enforced my confidence in my ability to communicate with other professionals within the field of Cyber security.

A lingering ethical and social issue that surfaced during the research was the application of risk, as calculated from the geographical location of the IP. This is a risk related factor that may be criticised in the future, as the application of risk ratings to countries of origin may be considered an attempt to stain the reputation of the country in terms of its cyber security risk. This, however, is arguably not accurate, as we have established during the conclusions that many attackers may be using VPN technology to appear as though they were operating from that specific country.

As mentioned in the previous section the Covid-19 virus presented an unforeseen flood of social and professional issues that needed to be addressed in order to progress with the project. Due to a health condition, I was compelled to enter a state of self isolation due to the increased risk from potential symptoms. This was followed by a period in which the pandemic continued to spread and the whole country was placed in lock-down, this also meant that meetings with my supervisor became more difficult to organise and we would just have to grab a few words when we could. In spite of these difficulties, I applied practical adjustments as outlined in the prior section to counteract this change of circumstances. 