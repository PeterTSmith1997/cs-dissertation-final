\section{UI testing}

Although the UI is not a major part of the project, it is still important to check that the UI incorporates the core functionality required for the system. Therefore, the UI was tested as the software was being built. All components of the the UI were functional as they were implemented, however, when looking back over the software it was found that there were some errors in the spelling of words. Some text fields also needed to be reworded in order to aid the user experience. Formal user testing was not implemented due to concerns over participation and candidate correspondence during Covid-19. It was believed the only way to test the UI properly was to enroll both expert and non expert users to evaluate the UI efficiently. The potential for future testing of the UI shall be discussed during the conclusions section. As mentioned in prior chapters the UI was built in accordance with Nielson's principles and criticisms of this shall also be debated in the conclusion of this work. 

During the agile build methodology, small changes to the UI were made in order to improve the user experience. The changes that were implemented during the design and testing phase shall be laid out in this section in order to present the efforts that were made to encourage smooth human-computer interaction.

It was noted that whilst the button that is required to read the file is present, it does not clearly highlight how to display the data. It would have been useful to make this button clearer in some way, however, due to the time constraints of this project this would require extensive reworking of the program which the aforementioned time constraints did not allow. By adding the words \textit{start here} to the button; this highlighted the button's presence and provided clear instructions on how to start the process.

Many of the Text fields in the UI were altered during the implementation of testing as it was believed that the wording may have been counter intuitive to non expert users. The header 'page' was changed to 'page accessed' to reflect a more self explanatory notation for non expert users. The second search box was removed as it was deemed to be superfluous and may have lead to confusion for users. The changes described in this section can be found in Appendix I.